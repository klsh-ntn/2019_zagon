\documentclass[12pt]{article} % размер шрифта

\usepackage{tikz} % картинки в tikz
\usepackage{microtype} % свешивание пунктуации

\usepackage{array} % для столбцов фиксированной ширины

\usepackage{indentfirst} % отступ в первом параграфе

\usepackage{multicol} % текст в несколько колонок
\usepackage{verbatim}

\usepackage{sectsty} % для центрирования названий частей
\allsectionsfont{\centering} % приказываем центрировать все sections

\usepackage{amsmath, amssymb} % куча стандартных математических плюшек

\usepackage[top=2cm, left=1.5cm, right=1.5cm, bottom=2cm]{geometry} % размер текста на странице

\usepackage{lastpage} % чтобы узнать номер последней страницы

\usepackage{enumitem} % дополнительные плюшки для списков
%  например \begin{enumerate}[resume] позволяет продолжить нумерацию в новом списке
\usepackage{caption} % подписи к картинкам без плавающего окружения figure


\usepackage{fancyhdr} % весёлые колонтитулы
\pagestyle{fancy}
\lhead{Загоночная по комплексным числам}
\chead{НТН, сквозной, У1}
\rhead{КЛШ-2019}
\lfoot{}
\cfoot{}
\rfoot{\thepage/\pageref{LastPage}}
\renewcommand{\headrulewidth}{0.4pt}
\renewcommand{\footrulewidth}{0.4pt}



\usepackage{todonotes} % для вставки в документ заметок о том, что осталось сделать
% \todo{Здесь надо коэффициенты исправить}
% \missingfigure{Здесь будет картина Последний день Помпеи}
% команда \listoftodos — печатает все поставленные \todo'шки

\usepackage{booktabs} % красивые таблицы
% заповеди из документации:
% 1. Не используйте вертикальные линии
% 2. Не используйте двойные линии
% 3. Единицы измерения помещайте в шапку таблицы
% 4. Не сокращайте .1 вместо 0.1
% 5. Повторяющееся значение повторяйте, а не говорите "то же"

\usepackage{fontspec} % поддержка разных шрифтов
\usepackage{polyglossia} % поддержка разных языков

\setmainlanguage{russian}
\setotherlanguages{english}

\setmainfont{Linux Libertine O} % выбираем шрифт
% можно также попробовать Helvetica, Arial, Cambria и т.Д.

% чтобы использовать шрифт Linux Libertine на личном компе,
% его надо предварительно скачать по ссылке
% http://www.linuxlibertine.org/

\newfontfamily{\cyrillicfonttt}{Linux Libertine O}
% пояснение зачем нужно шаманство с \newfontfamily
% http://tex.stackexchange.com/questions/91507/

\AddEnumerateCounter{\asbuk}{\russian@alph}{щ} % для списков с русскими буквами
\setlist[enumerate, 2]{label=\asbuk*),ref=\asbuk*} % списки уровня 2 будут буквами а) б) ...

%% эконометрические и вероятностные сокращения
\DeclareMathOperator{\Cov}{Cov}
\DeclareMathOperator{\Corr}{Corr}
\DeclareMathOperator{\Var}{Var}
\DeclareMathOperator{\E}{\mathbb{E}}
\def \hb{\hat{\beta}}
\def \hs{\hat{\sigma}}
\def \htheta{\hat{\theta}}
\def \s{\sigma}
\def \hy{\hat{y}}
\def \hY{\hat{Y}}
\def \v1{\vec{1}}
\def \e{\varepsilon}
\def \he{\hat{\e}}
\def \z{z}
\def \hVar{\widehat{\Var}}
\def \hCorr{\widehat{\Corr}}
\def \hCov{\widehat{\Cov}}
\def \cN{\mathcal{N}}
\let\P\relax
\DeclareMathOperator{\P}{\mathbb{P}}
\let\Re\relax
\DeclareMathOperator{\Re}{Re}



\begin{document}
\begin{enumerate}
    \item Подели и умножь комплексные числа и кватернионы: 
    \begin{multicols}{2}
    \begin{enumerate}
        \item $(1+3i)(2-5i)$ 
        \item $(3+5i)/(3+4i)$
        \item $(1+2i+3j+4k)(4+3i+2j+k)$
        \item $(2+5i+4k)/(-3+4j)$.
    \end{enumerate}
    \end{multicols}
    \item Черепаха стартует в точке $0$. В первую минуту она движется со скоростью один километр в минуту.
    Каждую последующую минуту она поворачивает на 60 градусов по часовой стрелке и увеличивает свою скорость в два раза. 
    Где черепаха окажется через час?
    \item Реши в комплексных числах уравнение $z^6 = -64$.
    \item Нарисуй множество $A =\{ \Re z = 3 \}$ и его образ $f(A)$ для функции $f(z) = 1/\bar z$.
    \item Рассмотри произвольный четырёхугольник $ABCD$. 
    С помощью комплексных чисел (или иначе) найди отношение суммы квадратов диагоналей к сумме квадратов средних линий.

\end{enumerate}

\newpage
\lhead{Загоночная по правдюкам и лжецам}
\chead{НОН, модуль 1, У3}

\begin{enumerate}
    \item Все жители острова либо правдюки, либо лжецы. Путешественник встретил пятерых аборигенов. 
    На его вопрос: «Сколько среди вас правдюков?» первый ответил: «Ни одного!», а двое других ответили: «Один». 
    Что ответили остальные?

    \item Альберт и Бернард только что познакомились с Шерил. Они хотят знать, когда у неё день рождения. 
    Шерил предложила им десять возможных дат: 15 мая, 16 мая, 19 мая, 17 июня, 18 июня, 14 июля, 16 июля, 14 августа, 15 августа и 17 августа. 
    Затем Шерил сказала Альберту месяц своего рождения, а Бернарду — день. После этого состоялся диалог.
    
    Альберт: Я не знаю, когда у Шерил день рождения, но я знаю, что Бернард тоже не знает.
    
    Бернард: Поначалу я не знал, когда у Шерил день рождения, но знаю теперь.
    
    Альберт: Теперь я тоже знаю, когда у Шерил день рождения. 
    
    Когда у Шерил день рождения?
    
  
    %\item На острове живут три Ученика: зеленоглазые Андрей и Борис и кареглазый Владимир. А ещё на острове живёт Гуру.
    %Ученики не общаются между собой и ни один из них не знает свой цвет глаз.
    %Однажды Гуру собрала Учеников вместе и объявила им: «Хотя бы один из вас — зеленоглазый».
    %Затем она спрашивает по очереди каждого из учеников (Андрея, Бориса, Вову, потом снова Андрея и так далее): 
    %«Знаешь ли ты свой цвет глаз?».
    
    %Что будут отвечать Ученики?


\end{enumerate}


\newpage
\lhead{Загоночная по прогнозированию}
\chead{НОН, модуль 2, У3}

Кот Матроскин записал надои коровы Мурки за последние четыре месяца: $20$, $30$, $30$, $40$. 
Построй прогноз надоев на один и два шага вперёд с помощью:

\begin{enumerate}
    \item наивного алгоритма;
    \item модели ETS(ANN) с параметрами $\alpha = 0.5$, $\ell_0 = 30$;
    \item модели ETS(AAN) с параметрами $\alpha = 0.5$, $\ell_0 = 20$, $\beta = 0.1$, $b_0 = 6$.
\end{enumerate}


Полезные уравнения:

\[
\begin{cases}
y_t = \ell_{t-1} + \varepsilon_t \\
\ell_t = \ell_{t-1} + \alpha \varepsilon_t \\
\hat y_{t+h} = \ell_t \\
\end{cases} \quad
\begin{cases}
y_t = \ell_{t-1} + b_{t-1} + \varepsilon_t \\
\ell_t = \ell_{t-1} + b_{t-1} + \alpha \varepsilon_t \\
b_t = b_{t-1} + \beta \varepsilon_t \\
\hat y_{t+h} = \ell_t + h b_t \\
\end{cases}    
\]

Подсказка: из первого уравнения системы можно выразить $\varepsilon_t$ и подставить его в остальные :)

\end{document}
